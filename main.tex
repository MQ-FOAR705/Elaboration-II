\documentclass{article}
\usepackage[utf8]{inputenc}

\title{Elaboration II}
\author{Jeremy Amin}
\date{August 2019}

\begin{document}


\maketitle

\tableofcontents

\section{Task Outline}

\noindent
In Scoping II I identified two patterns I wish to make easier. There were:

\begin{itemize}
    \item Finding specific words and sentences within and across different papers in an efficient manner
    \item Collecting and organizing papers into a single bibliographic database in such a way that I can search for papers based on a theme.

\end{itemize}

\noindent
Depending on the software I can find, I may modify the second bullet point to make it a simpler goal for my PoC. I will also include a referencing goal as a part of my PoC.

\begin{itemize}
    \item Find software which will serve as a reference database and which will do the referencing work for me 
\end{itemize}


\noindent
I notice that I did not make the Algorithm part of Scoping II detailed enough. It seems I misunderstood the difference between it and the Decomposition section. In the Elaboration process section I rectify this mistake.


\section{Potential Software to solve my problems}

\subsection{Voyant} 

Voyant seems like a good program for word and theme searches. This will solve some of the problems I identified in Scoping.

\subsection{Zotero}
Zotero seems like a sufficient program for referencing and as a bibliographic database. Alternatives include EndNote and Mendeley.

\subsection{Excel}

I think Excel will work well in conjunction with Voyant and Zotero as a way to thematically organise papers and other resources in a more visually simple and user friendly style. This is more for myself as a user on the GUI end of the research process.

\subsection{OverLeaf}

Document writing with more control over the formatting etc. compared to Word or Google Documents. Also enables version control in a seamless way.

\subsection{GitHub}

Version control website.

\section{Elaboration process}

The steps I need to take to test the software listed above are as follows:

\begin{itemize}
    \item Check the referencing which online databases offer (eg Multi Search at MQ Library website) and see how they compare to the way Zotero, EndNote, and Mendeley handle referencing.
    \item Upload multiple PDF's and other documents to Voyant and test how it handles word searches and thematic searches.
    \item Test Excel as a bibliographic database and compare it to Zotero, Mendeley, and EndNote.
    \item Test how easily I can manage Voyant, Zotero, and Excel with a word processing software such as Google Documents and OverLeaf.
\end{itemize}

\section{Software Testing}

I decided to only test Voyant and Zotero for the moment as Brian suggested I need to change up my PoC in order to attain a better overall grade for the unit and to create a more useful algorithm for my future as a philosophy academic.

\subsection{Voyant}

\begin{itemize}
    \item Uploaded 12 PDF documents which I know have a similar theme and word use and argument style
    \item Voyant summarised the themes and frequent word and phrase use to my satisfaction. I can see how this will be useful at quickly and efficiently filtering documents which will be useful for me.
    \end{itemize}
    
\subsection{Zotero}
\begin{itemize}
    \item I have uploaded over 30 references from multiple locations onto Zotero.
    \item Using the citation and bibliography functions has solved the referencing problem I outlined in Scoping.
\end{itemize}

\section{Revision}

I had a chat with Brian and have decided that I may need to change my entire PoC if I am to get a higher mark for this unit. I am thinking of working on a PoC which will allow for version controlling my supervisor's feedback during my thesis and outputting the documentation into a supervisor friendly format (probably in a Word or PDF format). This will allow me to have achieve two things. 

First it will provide a rigorous version controlled history of my feedback and actions based on the feedback. Secondly, it will automatically generate a reader friendly version of the report of my actions based on my supervisor's feedback. My software of choice will most likely be a combination of Overleaf, GitHub, and cloudstor. I may also be using something like Terminal (Shell) on my Mac Book to allow for pipe-lining the text from the Overleaf document into a Word document and perhaps as a PDF as well.

\section{Software chosen}

For my first PoC I would be using Voyant and Zotero to solve my problems of oganising my papers and searching for themes in the filtration process when selecting papers and other resources.

For my potential updated PoC I would be using at least OverLeaf, GitHub, Shell, cloudstor, and either Microsoft Word, Google Docs, or LibreOffice.



\end{document}

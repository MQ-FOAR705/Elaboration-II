\documentclass{article}
\usepackage[utf8]{inputenc}

\title{Elaboration II}
\author{Jeremy Amin}
\date{August 2019}

\begin{document}


\maketitle

\tableofcontents

\section{Instructions}

Test the technologies proposed in the previous elaboration activity to see if they perform as expected and could plausibly solve the pains (or produce the gains). Record your tests and results in your Learning Journal. These tests should be targeted and limited in scope. For example, if you are considering software that requires significant customisation, you might test out the closest existing customisation rather than trying to do your own customisation now.

Revise your Elaboration Plan and continue testing based on the outcomes of the prior tests until you have a set of technologies (tools and techniques for using the tools) that passes your tests, and offers a plausible solution which addresses a problem (or delivers a gain) defined in your scoping exercise. The ultimate outcome of this assignment is to have confidence in the tools and techniques you plan to use to solve one of the problems articulated by your scoping exercise. 

Remember that a technology is both a tool and the techniques needed to use the tool. Elaboration is an attempt to see if a tool can solve part of your problem (perhaps a decomposed step or two, perhaps part of a step) with an eye towards being able to combine the tools and techniques for using the tools in the rest of the semester. 

Now would be a good time to try your new knowledge of version control using GitHub. For best results, commit after each test.

\section{Software Testing}

\section{Revision}

\section{Software chosen}

\end{document}
